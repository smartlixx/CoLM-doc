\chapter*{前~言}
\addcontentsline{toc}{chapter}{前言}
\markboth{前言}{前言}
\echapter*{Preface}
%\addengcontents{chapter}{Foreword}

地球是作为一个系统在运行着。地球系统将大气、陆地(含陆地生物圈)、海洋(含海洋生物圈、海冰)、冰冻圈(大陆冰原)和岩石圈视为一个整体,由一系列相互作用过程(包括系统各组分之间的相互作用,物理、化学和生物三大基本过程的相互作用,以及人类与地球的相互作用)联系起来的复杂非线性的多重耦合系统。

地球系统模式(ESMs)是描述这一耦合系统的数学物理模式,旨在定量理解控制整个地球系统相互作用着的物理、化学和生物过程,人类活动影响与反馈,作为一个复杂、自适应性系统的结构、功能和运行机制。ESMs是地球系统预测的基本工具或唯一手段。

与传统物理气候系统模式相比,ESMs还包括元素(碳:氮:磷等)循环、陆地和海洋生态系统及生物地球化学过程、大气化学过程、以及自然和人为干扰。ESMs已超越气候模式对大气、陆地和海洋状态的物理描述,将气候预测扩展到更全面的地球系统预测,包括生态系统、水文水资源、人类活动(农林牧业、水利、城市和交通等)的影响与反馈等。

陆面过程模式作为数值天气预报模式/气候模式/ESMs的重要分系统模式。自20世纪60年代末70年代初始,已由简单“水桶”水分平衡和简单能量平衡陆面参数化方案,到当前的包含陆面物理、水文、植被生理生态、生物地球化学和人类活动的陆面过程模式,陆面过程模式的内涵和外延已全面扩充,内容已得到极大的丰富和发展。

一方面,它的一些重要的关键过程数学建模问题,例如,“冠层/积雪/城市辐射传输过程”,“水分相变过程”,“土壤水力学过程-可变饱和土壤水运动算法”,“产汇流水文过程:降水-冠层截流-土壤表面径流-坡面流-侧向流与基流(地下水流)-湿地湖泊河汇流-河道径流-人工水库-调水(包括地下水)-灌溉与排涝-工业与生活用水”,“植被光合作用与气孔导度关系,光合作用-水分关系”,“碳循环:包括总初级生产、自养呼吸、落叶、异养呼吸和野火等组成部分-植物和土壤碳储存池-其他生物地球化学通量包括沙尘、野火化学排放、生物挥发性有机化合物、活性氮循环以及湿地甲烷排放”,“通过考虑植被类型、叶面积、叶片上气孔以及碳和氮库来表示陆地生态系统”,以及“从叶片到植物冠层、从生态系统到景观再到生物群落的空间尺度问题;对当前环境条件的即时生理响应(如气孔导度、光合作用和呼吸,季节性落叶现象以及枯萎),在数十年和数百年时间尺度上的生态系统结构和生物地理对自然干扰、人为干扰和气候变化响应的时间尺度问题”等等,已逐步揭示清楚和数学建模。

另一方面,开展了一系列大规模的现场实验,例如,FIFE、BOREAS、LBA和中国青藏高原、黑河流域等项目,运用涡度相关技术精确测量生物圈与大气之间碳、水和能量的循环过程的通量观测网络FLUXNET等,为我们提供了深入理解陆面水、能量和生态系统功能的基础,提供了验证陆-气方案中所涉及的过程模型和尺度假设,这些实验还推动了将卫星数据解译为模式所需的全球地表参数集的发展。基于实地调研普查和卫星观测反演研制了全球高分辨率地形高程、土地覆盖/土地利用、土壤属性、植被属性、农业耕作、城市结构和功能、河流湖泊和冰川冻土等巨量数据集。得益于能严格解释辐射并将其解译为有用生物物理量的技术和方法的进步,从而使卫星遥感能提供全球陆地表面参数场最可行、一致和准确方法和数据集。卫星遥感数据已扮演至关重要的角色,成为可获取和可用于以改进和验证陆面模式的全球数据。正因为数据的不断丰富,极大推动陆面模式的发展,实现从不可能到可能。

再一方面,它和其它学科的交叉、渗透,包括气象学、水文学、植物生理生态学、生物地球化学、农学、卫星遥感应用、全球变化、应用数学、高性能计算等,它集成不同学科的优秀成果,提供了深入整合领域和丰富研究工具和方法,已为陆面过程建模工作注入勃勃生机和崭新内容。

我们于上世纪90年代初,开展研制服务于大气环流模式(AGCM)的陆面过程模式,1994年完成中国科学院大气物理研究所陆面过程模式(IAP LSM version 1994)研制,并实现与IAP-AGCM的耦合运行。1997年末启动研制通用陆面模式(The Common Land Model, CoLM),至此,已迭代了四个版本(即,CLM 1.0, CoLM 2004, CoLM 2014, CoLM 2024),十年发布一个版本。通用“Common”的最初旨意是为气候模式科学家们提供一个集成研究平台。CoLM 2024实现了真正字面意义上的“通用”。它可广泛应用于数值天气预报/气候预测、水文水资源、生态环境、城市、农林牧等行业的科学研究和精细化业务,适用于多尺度(\textasciitilde1米 至\textasciitilde100公里)应用。

地球系统模式开启了地球系统预测新时代,其中陆面模式的内涵和外延的全面扩充,大大拓展了ESMs的功能。

\textit{1、在数值天气/气候模拟研究中的作用}

天气/气候系统模式,将陆地视为一个物质和能量流动的物理系统,通过多种物理化学生命过程在不同空间时间尺度上影响天气/气候。从天气/气候模式工作者的角度来看,陆面过程模式在陆-气物质和能量通量的计算中起着基础性作用,尽可能提供准确的陆-气通量。陆-气间物质和能量通量对大气温度、水汽、降水、云、辐射以及环流等大气过程和状态产生重要影响。

陆-气垂向之间,水汽从陆面输送到大气底层。随后,大气中的水汽通过大尺度风和垂直扩散或对流进一步混合并向空间扩散。在水汽浓度超过饱和的气柱层内形成云和降水。这种过量的水汽要么来自大尺度上升运动,要么由湿对流产生。地表通量变化也可以对边界层云产生反馈作用,与深对流云一起,会改变地表能量通量。此外,地表加热的空间结构受到地形分布的制约,从而形成积云对流集中的区域。由于云凝结核种类的差异,陆地上云的形成过程可能与海洋上的云有很大不同。

不同尺度的陆地非连续性景观之间(海-陆、湖-陆、山坡-山谷、山地-平原、城-乡、灌溉-非灌溉农田等等)动力和热力差异会引发不同尺度的大气运动。不同尺度的环流如海风、湖风、山谷风等中小尺度环流,乃至大陆尺度或全球尺度大气环流。

陆地属性、状态及其变化,对大气在局地、区域和全球尺度上的直接和间接影响,已有大量的论述,粗糙度、反照率、土壤湿度、植被、积雪等对天气/气候的直接和间接,局地到全球的影响已有充分的研究。不同尺度的陆地属性、状态及其变化,包括自然因素(极端天气/气候、水文、生态、地质事件导致地表景观和属性的变化:洪涝、高温热浪、火灾、荒漠化、沼泽化、山崩地裂等)和人为因素(人类为了生存和美好生活行动:农林牧、城市建设/管理/生产生活、交通、水文水资源工程、绿水青山或生态环境工程等等)导致的改变,对不同尺度(时间:局地至全球,空间:瞬时至百年)大气过程的影响(由陆面模式输入到大气,对不同尺度大气过程的影响),反馈(由大气模式输入陆面,导致陆面的变化),及循环反馈直至平衡或跃级到新的平衡态(例如,高温、干旱、多雨、低温冰冻等天气/气候极端事件成为常态等),以及演变过程中的突变或极端事件。

正因为精细化的陆面模式,为诸多方面的陆-气相互作用研究提供了科学方法和工具。

\textit{2、在水文/水资源管理模拟研究中的作用}

基于物理原理的水文过程模式是深入理解、监测和预测水循环的重要工具。陆面模式水文过程建模集成了近年来水文学领域的最新进展,水文学家深度参与了模式的研发。陆面模式精细描述了对植被冠层截留过程、积雪和土壤水文过程(垂向、侧向流动和地下水),产、汇流过程,湿地、湖泊、水库、河道径流过程,水文连通性及多尺度空间变异性,人类活动等关键水文过程,以及全面的水文模式基准验证(流域或关键带测站数据)与系统性地严格评估。

陆面模式适用不同空间尺度(\textasciitilde1米 至\textasciitilde100公里)水文过程模拟。其中,基于流域水文响应单元的水文过程建模,明确解决了次流域尺度的地形结构方面存在问题,显著改善了次流域尺度上的水分、能量及生物地球化学过程与通量模拟,实现在空间和水文学过程要素之间进行综合,提升模式的物理真实性和普适性,这些努力已超越传统水文学中采用的方法。

由于地域与过程的复杂性、广泛的模型需求,以及众多建模群体等,导致当前水文领域模型数量和种类繁多。过多水文模型,可能会导致研究资源浪费和低效率。基于服务于天气/气候/地球系统模式的陆面模式这一系统集成平台,开展水文模型研发、机理研究与业务应用等,将极大推动水文过程建模。

\textit{3、在生态系统和农业生产与管理模拟中的作用}

气候模式迈向地球系统模式(ESMs)的得益于陆面分系统模式的重大进展。正因为此,ESMs可以模拟陆地生态系统以及生物地球化学循环,为相关于气候过程的生态研究提供了一个通用框架,包括:物候、生长季节长度和群落组成,水利用效率和生产力,野火、昆虫、极端事件和人类活动引起的干扰等;以及脆弱性分析、影响和适应性分析,气候变化缓解措施等。ESMs使我们可以超越传统对大气状态的物理描述,转而可关注与社会相关的问题。例如,火灾风险、栖息地丧失、水资源可用性以及作物、木材产量等问题;干旱、热浪、不稳定且强烈的降水分布、风暴和洪水,$\rm CO_2$、$\rm O_3$,土壤侵蚀、土壤盐碱化,用水需求和可用性等影响农业生产力及稳定性和农民生计问题。ESMs不仅提供了评估未来气候变化影响的方法,还能确定气候变化对生物圈和农业生产与管理的影响,以及生态系统固碳能力和农业适应气候变化策略有效性的评估与预估提供基础理论与应用平台。

本报告是一系列介绍了通用陆面模式2024版本(The Common Land Model version 2024, CoLM 2024)的第一册,为用户提供了 CoLM 的全面详尽的描述,包括其所有的陆面物理、化学、水文、生态和人类活动等过程的数学建模的科学和算法方法。此技术说明覆盖了所有在CoLM版本2024之前发布的版本,并将随着新版本发布及功能添加进行更新。

本报告由八部分组成,共分三十章。第一部分为引言,介绍通用陆面模式CoLM的由来及其版本发展历史,并简要介绍CoLM 2024的新特性。第二部分为模式构架与基础数据,讲明模型的总体结构与计算框架,特别是本版本新增加的非结构网格、流域单元网格以及植物功能型次网格和植物群落次网格结构;另外,对模型所需的地表覆盖、土壤、植被、水文、城市等数据进行简要介绍。第三部分为地表通量方案,包括地表辐射通量计算、地表湍流通量计算、光合作用与气孔导度计算、植被水力模式以及降水与地表的能量交换。第四部分介绍植被冠层、积雪和土壤温度计算方案。第五部分为植被冠层、积雪和土壤水分计算方案,包括植被冠层雨水截留过程以及积雪、土壤水分垂直运动过程。第六部分为水文过程,包括耦合径流模型的产流、汇流过程及侧向流的模拟;对湖泊模式、冰川模式和湿地模式进行介绍。第七部分为生物地球化学循环过程,说明植被和土壤碳库结构,植被和土壤凋落物的生物地球化学循环过程及其预热加速功能,并描述火灾模块。第八部分与人类活动相关,涵盖对城市模块、作物模块、水库模块和土地利用土地覆盖变化模块的描述。

CoLM由中山大学大气科学学院戴永久研究团队开发和维护。CoLM属于开源系统,我们欢迎任何个人或实体均可在任何目的下无需支付任何费用下载和使用。最新的模式程序和文档可以通过以下网址获取:~\url{https://github.com/CoLM-SYSU/CoLM202X}(源代码),\url{https://github.com/CoLM-SYSU/CoLM-doc}(科学与技术报告)和~\url{https://github.com/CoLM-SYSU/CoLM-UsersGuide}(用户手册)。
